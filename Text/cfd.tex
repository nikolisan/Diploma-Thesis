Στο κεφάλαιο αυτό παρουσιάζεται η μοντελοποίηση της κυκλοφορίας σε έναν εξιδανικευμένο κόλπο, όπου το ένα άκρο του είναι ανοιχτό (\lat{open-sea-boundary}), ενώ τα υπόλοιπα κλειστά (στεριά).

Ο κόλπος έχει εξιναδικευτεί ώστε να έχει κάτοψη ορθογωνίου, με σταθερό βάθος και άτριβο πυθμένα. Το μοντέλο περιγράφει τη διάδοση ενός παλιρροϊκού κύματος από την ανοιχτή θάλασσα και το κλειστό άκρο, στο οποίο προσπίπτει το κύμα, προκαλεί τέλεια ανάκλαση.
\subsection{Παλίρροια και παλιρροϊκή κυκλοφορία}
Το φαινόμενο της παλίρροιας αναφέρεται στην περιοδική κίνηση του νερού της θάλασσας και οφείλεται στην διαφορά των ελκτικών δυνάμεων του Ηλίου και της Σελήνης, στις διάφορες φάσεις της.
Ως παλίρροια αναφέρεται στην περιοδική ανύψωση και ταπείνωση της στάθμης της θάλασσας και συνοδεύεται από το παλλιροϊκό ρεύμα, δηλαδή την οριζόντια κίνηση του νερού. Και οι δύο κινήσεις αποτελούν μέρη του ίδιου φαινομένου.

Η περιοδική ανύψωση και ταπείνωση της στάθμης της θαλάσσας είναι επί της ουσίας ένα μακρύ κύμα το οποίο διαδίδεται. Η περίοδος ενός τέτοιου κύματος είναι συγκρίσιμη με το χρόνο περιστροφής της γης, αφού η παλίρροια μπορεί να έχει περίοδο 6, 12, ακόμα και 24 ώρες, οπότε επηρεάζεται από αυτή. Για αυτό στις εξισώσεις ορμής είναι απαραίτητη η προσθήκη του όρου της δύναμης \cor.

Ένα μακρύ κύμα περιγράφεται από την εξίσωση 
\begin{equation}
    η = \dfrac{H}{2}\cos{(kx-ωt)}
\end{equation}
όπου $Η$ το πλάτος κύματος, $k=\dfrac{2π}{L}$ o κυματαριθμός, $ω=\dfrac{2π}{T}$ η κυκλική συχνότητα, $L$ το μήκος κύματος και $T$ η περίοδος.

\subsection{Εξισώσεις ρηχού στρώματος - \lat{SWE}}
Πολλοί τύποι ροών μπορούν να χαρακτηριστούν ως "ροές ρηχού στρώματος". Με τον όρο ροή δεν είναι απαραίτητη η παραπομπή μόνο σε ροή ύδατος. Ροές όπως η κυκλοφορία ατμοσφαιρικών μαζών, η παλίρροια, πλημυρικά ρέματα, η παράκτια κυκλοφορία, τα \lat{tsunamis} κ.ά., μπορούν να περιγραφούν με τις εξισώσεις ρηχού στρώματος. Το βασικό χαρακτηριστικό τέτοιων ροών είναι ότι η κλίμακα του βάθους είναι πολύ μικρότερη από την οριζόντια χαρακτηριστική κλίμακα.
\begin{equation}
    \dfrac{h}{L} \ll 1
\end{equation}

Επομένως το αν θα εμπίπτει μία ροή στην κατηγορία εξισώσεων ρηχού στρώματος έχει σχέση με τις γεωμετρικές διαστάσεις της ίδιας της ροής και όχι του ρευστού όγκου που εξετάζεται. Για παράδειγμα η παλιρροϊκή ροή σε έναν ωκεανό περιγράφεται από τις εξισώσεις ρηχού στρώματος, όμως η ανεμογενής κυκλοφορία σε έναν ταμιευτήρα όχι, παρόλο που το βάθος του ταμιευτήρα είναι πολύ μικρό.

Στο μοντέλο του ρηχού στρώματος υπεισέρχεται η βασική απλοποιήση ότι η κατακόρυφη συνιστώσα της ταχύτητας, οπότε και της επιτάχυνσης είναι αμελητέες. Έτσι, η κατακόρυφη εξίσωση ορμής αποτελείται από υδροστατική ισορροπία.

Το σύστημα των διαφορικών εξισώσεων που περιγράφουν το πρόβλημα είναι οι εξισώσεις ορμής \lat{Navier - Stokes} και η εξίσωση συνέχειας. Ύστερα από ανάλυση της κλίμακας κάθε όρου των εξισώσεων και απαλοιφής των ασήμαντων, λόγω κλίμακας, όρων οι εξισώσεις ορμής για τους άξονες $x$, $y$, $z$ είναι αντίστοιχα
\begin{align}
    \dfrac{\partial{u}}{\partial{t}} + u\dfrac{\partial{u}}{\partial{x}} + v\dfrac{\partial{u}}{\partial{y}} + w\dfrac{\partial{u}}{\partial{z}} &= fv-\dfrac{1}{ρ}\dfrac{\partial{p}}{\partial{x}} + \dfrac{\partial}{\partial{z}}\left(ε_z\dfrac{\partial{u}}{\partial{z}}\right) \\
    \dfrac{\partial{v}}{\partial{t}} + u\dfrac{\partial{v}}{\partial{x}} + v\dfrac{\partial{v}}{\partial{y}} + w\dfrac{\partial{v}}{\partial{z}} &= fv-\dfrac{1}{ρ}\dfrac{\partial{p}}{\partial{y}} + \dfrac{\partial}{\partial{z}}\left(ε_z\dfrac{\partial{v}}{\partial{z}}\right) \\
    0 &= -\dfrac{1}{ρ}\dfrac{\partial{p}}{\partial{z}} - g
\end{align}
Το σύστημα συμπληρώνεται με την εξίσωση συνέχειας
\begin{equation}
    \dfrac{\partial{u}}{\partial{x}} + \dfrac{\partial{v}}{\partial{y}} + \dfrac{\partial{w}}{\partial{z}} = 0
\end{equation}

Η ύπαρξη ελεύθερης επιφάνειας του νερού εισάγει έναν ακόμα άγνωστο στο πρόβλημα της κυκλοφορίας. Η ανάπτυξη οποιουδήποτε ρεύματος συνεπάγεται τη μετατόπιση της ελεύθερης επιφάνειας από τη θέση ισορροπίας της, με αποτέλεσμα το πεδίο ορισμού της λύσης να είναι άγνωστο.

Ένας τρόπος αντιμετώσης αυτού του προβλήματος είναι να δημιουργήσουμε μία εξίσωση που να περιγράφει την θέση της ελεύθερης επιφάνειας κάθε χρονική στιγμή. Για το λόγο αυτό ολοκληρώνεται ως προς το βάθος η εξίσωση συνέχειας.

Έστω ότι με $ζ(t, x, y)$ συμβολίζεται το τοπικό βάθος της υδάτινης μάζας. Τότε η ολοκληρωμένη ως προς το βάθος εξίσωση συνέχειας δίνει
\begin{equation}
    \dfrac{\partial{ζ}}{\partial{t}} + \dfrac{\partial}{\partial{x}}\left(ζU\right) + \dfrac{\partial}{\partial{y}}\left(ζV\right) = 0 \label{eq:int-cont}
\end{equation}
όπου $U$ και $V$ οι μέσες ως προς το βάθος τιμές των $u$ και $v$ αντίστοιχα.

Ακόμα ολοκληρώνοντας την υδροστατική εξίσωση ως προς το βάθος, καταλήγουμε
\begin{align}
    -\dfrac{1}{ρ}\dfrac{\partial{p}}{\partial{x}} &= -g\dfrac{\partial{η}}{\partial{x}}-\dfrac{1}{ρ}\dfrac{\partial{p_α}}{\partial{x}} \\
    -\dfrac{1}{ρ}\dfrac{\partial{p}}{\partial{y}} &= -g\dfrac{\partial{η}}{\partial{y}}-\dfrac{1}{ρ}\dfrac{\partial{p_α}}{\partial{y}}
\end{align}

Μετά από τις κατάλληλες αντικαταστάσεις στο αρχικό σύστημα καταλήγουμε στις εξισώσεις ορμής, που μαζί με την ολοκληρωμένη ως προς το βάθος εξίσωση συνέχειας, αποτελούν τη βάση του μοντέλου ρηχού στρώματος
\begin{align}
    \dfrac{\partial{u}}{\partial{t}} + u\dfrac{\partial{u}}{\partial{x}} + v\dfrac{\partial{u}}{\partial{y}} + w\dfrac{\partial{u}}{\partial{z}} &= fv-g\dfrac{\partial{η}}{\partial{x}}-\dfrac{1}{ρ}\dfrac{\partial{p_α}}{\partial{x}} + \dfrac{\partial}{\partial{z}}\left(ε_z\dfrac{\partial{u}}{\partial{z}}\right) \label{eq:u-mom-non}\\
    \dfrac{\partial{v}}{\partial{t}} + u\dfrac{\partial{v}}{\partial{x}} + v\dfrac{\partial{v}}{\partial{y}} + w\dfrac{\partial{v}}{\partial{z}} &= fv-g\dfrac{\partial{η}}{\partial{y}}-\dfrac{1}{ρ}\dfrac{\partial{p_α}}{\partial{y}} + \dfrac{\partial}{\partial{z}}\left(ε_z\dfrac{\partial{v}}{\partial{z}}\right) \label{eq:v-mom-non}
\end{align}

\subsubsection{Τελικό μοντέλο}
Το μοντέλο που εξετάζεται στην παρούσα διπλωματική είναι αρκετά απλοποιημένο και δεν είναι απαραίτητη η χρήση όλων των όρων των εξισώσεων. Οι μη γραμμικοί όροι στις παραπάνω εξισώσεις μεταθέτουν μεν την ορμή, χωρίς όμως να μεταβάλλουν την ολική τιμή της. Καθώς η κυκλοφορία καθορίζεται από την ισορροπία των δυνάμεων, η μορφή της διατηρείται και στη γραμμικοποιημένη μορφή των εξισώσεων. Ακόμα, θεωρούμε ότι ο πυθμένας του κόλπου είναι άτριβος, η ατμοσφαιρική πίεση είναι σταθερή σε όλη την κάτοψη, καθώς και η εξέλιξη του φαινομένου γίνεται στο ίδιο γεωγραφικό πλάτος, με αποτέλεσμα η δύναμη \cor να μην επηρεάζει τη λύση.

Με βάση όλες αυτές τις παραδοχές, οι εξισώσεις \ref{eq:u-mom-non}, \ref{eq:v-mom-non} και \ref{eq:int-cont} γίνονται
\begin{align}
    \dfrac{\partial{u}}{\partial{t}} &= -g\dfrac{\partial{η}}{\partial{x}} \label{eq:u} \\
    \dfrac{\partial{v}}{\partial{t}} &= -g\dfrac{\partial{η}}{\partial{y}} \label{eq:v}
\end{align}
\begin{equation}
    \dfrac{\partial{η}}{\partial{t}} + h\dfrac{\partial{U}}{\partial{x}} + h\dfrac{\partial{V}}{\partial{y}} = 0 \label{eq:int-cont-new}
\end{equation}

Μετά από παραγώγιση των εξισώσεων \ref{eq:u} και \ref{eq:v} ως προς $x$ και $y$ αντίστοιχα και παραγώγιση της \ref{eq:int-cont-new} ως προς $t$, καταλήγουμε στην εξίσωση κύματος
\begin{equation}
    \dfrac{\partial^2{η}}{\partial{t}^2} = gh\left( \dfrac{\partial^2{η}}{\partial{x}^2} + \dfrac{\partial^2{η}}{\partial{y}^2} \right) \label{eq:wave}
\end{equation}
η οποία είναι μια υπερβολική διαφορική εξίσωση της μορφής $\dfrac{\partial^2{\vec{u}}}{\partial{t}^2}=c^2 \nabla^2 \vec{u}$, με $\vec{u} = \vec{u}(x_1, x_2, ... , x_n)$ βαθμωτή συνάρτηση της οποίας οι τιμές αποτελούν τη μετατόπιση ενός κύματος. Γίνεται αμέσως φανερό ότι η ταχύτητα του κύματος είναι 
\begin{equation}
    c=\sqrt{gh} \label{eq:celerity}
\end{equation}

\subsection{Διακριτοποίση των εξισώσεων}